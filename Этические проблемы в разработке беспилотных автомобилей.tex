\documentclass{article}
\usepackage[utf8]{inputenc}
\usepackage[russian]{babel}
\usepackage{hyperref}

\begin{document}

\title{Этические проблемы в разработке беспилотных автомобилей}

\author{Эрнисов Айдар Эрнисович}
\maketitle

\section{Введение}

Беспилотные автомобили, основанные на искусственном интеллекте, несут в себе обещание революционизировать транспорт, повысив безопасность, эффективность и удобство. Однако их внедрение сопряжено с рядом сложных этических вопросов, которые необходимо тщательно изучить и решить, прежде чем эти автомобили станут повсеместной реальностью.

\section{Тезисы}

\subsection{Безопасность и принятие рисков}

\begin{itemize}
    \item Главная идея: Обеспечение безопасности беспилотных автомобилей является первостепенной задачей. Это включает в себя разработку надежных систем ИИ, способных безопасно маневрировать в различных дорожных условиях, а также минимизацию рисков отказов оборудования и кибератак.
    
    \item Структура:
    \begin{itemize}
        \item Потенциальные угрозы безопасности:
        
        Кибервзломы: Хакеры могут проникнуть в системы компании или организации, чтобы получить доступ к конфиденциальной информации, воровать данные о клиентах или сотрудниках, а также нарушать работу систем.

        Вирусы и вредоносное ПО: Зловредное программное обеспечение может инфицировать компьютеры и сети, блокировать доступ к данным или украсть конфиденциальные сведения. В некоторых случаях вирусы могут привести к полной потере данных или недоступности систем.

        Физические аварии: Это могут быть различные чрезвычайные ситуации, такие как пожары, наводнения, а также повреждение оборудования из-за естественных бедствий или человеческих ошибок.

        Недоступность системы: Например, отказ сервера или сбой сети может привести к простою в работе, что может нанести ущерб бизнесу или организации.

        Неавторизованный доступ: Сотрудники или сторонние лица могут получить доступ к конфиденциальной информации или системам без разрешения, что может привести к утечкам данных или другим серьезным последствиям.

        Социальная инженерия: Злоумышленники могут использовать манипуляции и обман, чтобы получить доступ к конфиденциальной информации или учетным данным путем общения с сотрудниками или другими людьми внутри организации.

        Отказ в обслуживании (DoS) или распределенный отказ в обслуживании (DDoS): Атаки DoS и DDoS могут нацелиться на перегрузку серверов или сетей, делая их недоступными для легальных пользователей.

        Уязвимости в программном обеспечении: Наличие уязвимостей в программном обеспечении может быть использовано злоумышленниками для взлома системы или получения несанкционированного доступа.
        \item Уровень безопасности беспилотных автомобилей с традиционными автомобилями.

        Техническая надежность: Беспилотные автомобили используют компьютерные системы, датчики и алгоритмы для восприятия окружающей среды и принятия решений на дороге. Уровень надежности и стабильности этих систем играет решающую роль в безопасности автомобиля.

        Кибербезопасность: Беспилотные автомобили могут стать мишенями для кибератак из-за своей связности и зависимости от программного обеспечения. Защита от киберугроз и обеспечение безопасности сетевых соединений играют критическую роль в предотвращении возможных атак.

        Алгоритмы и искусственный интеллект: Решения, принимаемые беспилотными автомобилями на основе алгоритмов и искусственного интеллекта, должны быть надежными и безопасными. Это включает в себя способность распознавать объекты на дороге, предсказывать их поведение и принимать безопасные решения на основе этой информации.

        Инфраструктура и законодательство: Уровень безопасности беспилотных автомобилей также зависит от инфраструктуры дорог и среды, в которой они работают, а также от законодательства и регулирования, устанавливающих стандарты и требования к безопасности.

        
        \item Этические вопросы, связанные с тестированием и развертыванием беспилотных автомобилей на дорогах общего пользования.

        Безопасность пассажиров и пешеходов: Один из главных этических вопросов заключается в том, насколько безопасными должны быть беспилотные автомобили для пассажиров и окружающих людей на дороге. Как балансировать между стремлением к максимальной безопасности и возможными рисками для жизни и здоровья?

        Принципы принятия решений: Беспилотные автомобили могут сталкиваться с ситуациями, когда необходимо принимать этические решения, например, выбирать между действиями, которые могут привести к ущербу пассажирам или окружающим людям. Какие принципы должны руководить этими решениями?

        Ответственность и правовые аспекты: Кто несет ответственность в случае аварии с участием беспилотного автомобиля? Какие правовые и этические нормы должны регулировать деятельность производителей, владельцев и операторов беспилотных автомобилей?

        Приватность и сбор данных: Беспилотные автомобили могут собирать большие объемы данных о своих пассажирах и окружающей среде. Как обеспечить защиту приватности этих данных и предотвратить их злоупотребление?

        Вопросы занятости и экономические аспекты: Внедрение беспилотных автомобилей может повлиять на рабочие места в сфере водительского труда. Как обеспечить переобучение и переквалификацию работников, которых затронет автоматизация?

        Доступность и социальная справедливость: Как гарантировать, что беспилотные автомобили будут доступны для всех слоев населения, включая людей с ограниченными возможностями, и не усилят социальные неравенства?

        \item Этические вопросы, связанные с тестированием и развертыванием беспилотных автомобилей на дорогах общего пользования.

        Безопасность и риски: Какие уровни риска допустимы при тестировании беспилотных автомобилей на дорогах, и какие меры безопасности следует принять для защиты пассажиров, других участников дорожного движения и окружающей среды?

        Аварийные ситуации и принятие решений: Как беспилотные автомобили должны реагировать на аварийные ситуации, включая выбор между действиями, которые могут поставить под угрозу жизнь пассажиров или других людей на дороге?

        Приватность и сбор данных: Как обеспечить защиту личной информации пассажиров, собираемой и обрабатываемой беспилотными автомобилями, и какие правила должны регулировать сбор, хранение и использование этих данных?

        Обучение и ответственность: Как обеспечить, чтобы беспилотные автомобили были обучены на разнообразных сценариях на дороге и могли правильно реагировать на неожиданные ситуации? Кто несет ответственность в случае аварии или инцидента, связанного с беспилотным автомобилем?

        Экономические последствия: Какие будут экономические последствия внедрения беспилотных автомобилей для отрасли транспорта и связанных с ней секторов, включая занятость и владение транспортными средствами?

        Доступность и социальная справедливость: Как обеспечить доступность беспилотных автомобилей для всех слоев населения, включая людей с ограниченными возможностями, и предотвратить углубление социальных неравенств?

        Воздействие на окружающую среду: Каковы будут воздействия развертывания беспилотных автомобилей на окружающую среду, включая транспортные потоки, загрязнение и использование природных ресурсов?
    \end{itemize}
\end{itemize}


\subsection{Автономия и ответственность}

\begin{itemize}
    \item Главная идея:  Определить, кто несет ответственность в случае аварии с беспилотным автомобилем: производитель, разработчик программного обеспечения, владелец или пассажиры?
    
    \item Структура:
    \begin{itemize}
        \item Различные правовые и этические рамки для определения ответственности в автономных системах.

        Принцип ответственности за действия: Этот принцип предполагает, что лица или организации, создавшие, развернувшие или управляющие автономной системой, несут ответственность за ее действия и последствия. Это может включать в себя производителей, разработчиков, владельцев и операторов системы.

        Принцип обучения и обучаемости: Согласно этому принципу, ответственность может быть определена на основе того, была ли система должным образом обучена и протестирована на разнообразных сценариях, чтобы минимизировать риск возникновения нежелательных ситуаций.

        Принцип соблюдения правил и законов: Автономные системы должны соблюдать применимые законы и нормативные акты, и их ответственность может определяться в соответствии с этими правилами. Это может включать в себя соблюдение правил дорожного движения и других нормативных требований.

        Принцип прозрачности и учета: Подход, основанный на прозрачности и учете, предполагает, что ответственность за действия автономной системы должна быть ясно определена и документирована, чтобы обеспечить прозрачность и возможность учета.

        Принцип совместной ответственности: Согласно этому принципу, ответственность может быть распределена между несколькими сторонами, включая производителей, владельцев, операторов и регуляторов, в зависимости от их роли и вклада в функционирование автономной системы.

        Принцип общественного контроля и участия: Этот принцип предполагает, что общественность должна иметь возможность участвовать в процессе определения правовых и этических рамок для автономных систем и контролировать их деятельность в интересах общества.
        
        \item Вопросы прозрачности и подотчетности в алгоритмах принятия решений ИИ, используемых в беспилотных автомобилях.

        Понимание алгоритмов принятия решений: Какие алгоритмы используются для принятия решений беспилотными автомобилями, и как эти алгоритмы принимают во внимание различные факторы, такие как безопасность, эффективность и соответствие правилам дорожного движения?

        Прозрачность принятия решений: Насколько прозрачны и понятны принятые беспилотным автомобилем решения в различных ситуациях на дороге? Можно ли объяснить, почему система сделала тот или иной выбор в конкретной ситуации?

        Возможность интерпретации решений: Могут ли пользователи, регуляторы или независимые эксперты анализировать и интерпретировать принятые решения беспилотного автомобиля? Насколько эти решения могут быть обоснованы и оправданы?

        Отслеживаемость принятых решений: Каким образом система ведет учет принятых решений и их последствий, и какая информация доступна для последующего анализа и оценки?

        Ответственность за решения: Кто несет ответственность за принятые беспилотным автомобилем решения, и как это можно установить в случае возникновения проблем или аварийных ситуаций?

        Обучение и улучшение системы: Каким образом принимаемые решения используются для обучения и улучшения алгоритмов беспилотного автомобиля, и как этот процесс может быть контролируемым и прозрачным?
        
        \item Потенциальные проблемы, связанные с делегированием ответственности за управление транспортным средством машине.

        Технические сбои и неполадки: Беспилотные автомобили могут сталкиваться с техническими проблемами или сбоями, которые могут привести к неправильному функционированию системы и потенциально опасным ситуациям на дороге.

        Недостаток опыта и интуиции: В отличие от человеческих водителей, беспилотные автомобили не обладают опытом и интуицией, которые могут помочь им правильно реагировать на неожиданные ситуации на дороге.

        Ответственность в аварийных ситуациях: В случае аварии или нежелательного события, виновным может оказаться разработчик или производитель беспилотного автомобиля, что может привести к юридическим и финансовым последствиям.

        Проблемы взаимодействия с другими участниками дорожного движения: Беспилотные автомобили должны эффективно взаимодействовать с другими участниками дорожного движения, такими как пешеходы, велосипедисты и водители других автомобилей, что может быть вызвать проблемы из-за различий в поведении и взаимодействии.

        Этические дилеммы: Беспилотные автомобили могут сталкиваться с этическими дилеммами в случае неизбежных аварийных ситуаций, где необходимо выбирать между действиями, которые могут привести к ущербу для различных участников дорожного движения.

        Безопасность данных и кибербезопасность: Беспилотные автомобили могут стать мишенями для кибератак и взломов, что может привести к неправильному управлению автомобилем и опасным ситуациям на дороге.

        
    \end{itemize}
\end{itemize}


\subsection{Этика приоритетов и права пешеходов}

\begin{itemize}
   \item Главная идея: Как беспилотные автомобили должны принимать решения в сложных ситуациях, когда необходимо выбирать между причинением вреда пешеходам, пассажирам или другим участникам дорожного движения?

   \item Структура:

   \begin{itemize}
        \item Различные этические принципы, которые могут быть использованы для программирования беспилотных автомобилей (например, утилитаризм, деонтология).

        Утилитаризм: Этот принцип основан на максимизации общего блага или полезности. В контексте беспилотных автомобилей, это может означать выбор действий, которые минимизируют количество аварий и ущерба для всех участников дорожного движения.

        Деонтология: Деонтологический подход основан на идеях о правильных и неправильных действиях, независимо от их последствий. Это может включать в себя соблюдение правил дорожного движения и этических норм, даже если это приводит к нежелательным результатам.

        Виртуозная этика: Этот принцип фокусируется на развитии характера и навыков, которые способствуют принятию этических решений. Для беспилотных автомобилей это может означать обучение системы обнаруживать и реагировать на этические аспекты дорожного движения, такие как приоритет пешеходам на пешеходных переходах.

        Справедливость и солидарность: Этот принцип предполагает, что решения должны быть приняты с учетом справедливости и солидарности со всеми участниками дорожного движения, даже если это означает неудобства для некоторых.

        Принцип заботы и уважения к достоинству: Этот принцип подчеркивает важность уважения к человеческой жизни и достоинству всех участников дорожного движения, что может повлиять на решения, принимаемые беспилотным автомобилем в аварийных ситуациях.
        
        \item Проблемы справедливости и неравенства в алгоритмах принятия решений ИИ, используемых в беспилотных автомобилях.

        Систематические предвзятости: Алгоритмы машинного обучения, используемые в беспилотных автомобилях, могут быть подвержены систематическим предвзятостям из-за нерепрезентативности тренировочных данных. Например, если данные о безопасности дорожного движения неравномерно представлены для различных групп населения (например, по полу, возрасту или расовой принадлежности), это может привести к неравному воздействию алгоритмов на разные группы.

        Проблемы сбалансированности при принятии решений: Алгоритмы беспилотных автомобилей могут сталкиваться с проблемами справедливого распределения рисков и ресурсов между различными участниками дорожного движения. Например, в ситуации, когда необходимо выбирать между действиями, которые могут поставить под угрозу жизнь водителя и жизнь пешехода, алгоритм должен учитывать этические и справедливые аспекты принятия решений.

        Прозрачность и объяснимость решений: Некоторые алгоритмы ИИ могут быть сложными и непонятными для обычных пользователей и даже для разработчиков. Это может создавать проблемы в объяснении и оправдании принятых решений, особенно в случае несчастных случаев или аварийных ситуаций.

        Ответственность и правовые аспекты: Вопросы ответственности и правового регулирования использования беспилотных автомобилей также могут вызывать проблемы справедливости и неравенства. Например, как распределить ответственность в случае аварии, вызванной действиями беспилотного автомобиля, могут возникнуть вопросы о справедливости и солидарности.

        \item Анализы потенциального влияния беспилотных автомобилей на права и безопасность пешеходов.

        Безопасность пешеходов: Беспилотные автомобили могут внести значительные улучшения в безопасность пешеходов на дороге, поскольку они обычно оснащены передовыми системами обнаружения и избегания столкновений, которые могут помочь предотвратить аварии с участием пешеходов.

        Взаимодействие с пешеходами: Взаимодействие между беспилотными автомобилями и пешеходами может потребовать дополнительного внимания. Например, беспилотные автомобили должны быть способны обнаруживать пешеходов на дороге и принимать безопасные решения при их приближении, учитывая их непредсказуемость и разнообразные поведенческие особенности.

        Право преимущества на пешеходных переходах: Беспилотные автомобили должны учитывать право пешеходов на переход дороги на пешеходных переходах. Это требует разработки алгоритмов и систем, которые гарантируют безопасное прохождение пешеходов через дорогу.

        Общественное пространство и уличная инфраструктура: Внедрение беспилотных автомобилей может привести к изменениям в уличной инфраструктуре и организации общественного пространства, чтобы учесть их взаимодействие с пешеходами. Например, могут быть созданы специальные зоны для безопасного движения беспилотных автомобилей и пешеходных зон.

        Обеспечение доступности для всех: Важно обеспечить, чтобы беспилотные автомобили были доступны для всех слоев населения, включая людей с ограниченными возможностями, чтобы не усугублять неравенство в доступности транспорта.

        Обучение и информирование пешеходов: Пешеходам также может потребоваться обучение и информирование о безопасном взаимодействии с беспилотными автомобилями, чтобы минимизировать риск возникновения конфликтных ситуаций на дороге.
    \end{itemize}

\end{itemize}


\subsection{Прозрачность и конфиденциальность данных}

\begin{itemize}
    \item Главная идея: Беспилотные автомобили будут собирать большие объемы данных о своих пассажирах, окружающей среде и других участниках дорожного движения. Как эти данные будут собираться, храниться и использоваться?
    
    \item Структура:
    \begin{itemize}
        \item Проблемы конфиденциальности и защиты данных, связанные с беспилотными автомобилями.

        Личные данные пользователей: Беспилотные автомобили могут собирать личные данные пользователей, такие как местоположение, путевые данные, предпочтения и привычки. Неправильное использование или утечка этих данных может привести к нарушению приватности и конфиденциальности пользователей.

        Системы навигации и картографии: Для своей работы беспилотные автомобили используют системы навигации и картографии, которые могут содержать чувствительные данные о маршрутах, местах и даже домах пользователей. Утечка этих данных может представлять угрозу для их безопасности.

        Данные датчиков и камер: Беспилотные автомобили обычно оснащены датчиками и камерами, которые собирают данные об окружающей среде и дорожной обстановке. Эти данные могут содержать информацию о других участниках дорожного движения и окружающей среде, и их утечка может нарушить их конфиденциальность.

        Коммуникационные сети и связь: Беспилотные автомобили могут использовать коммуникационные сети для связи с другими устройствами и системами, что создает риск перехвата и взлома данных, а также нежелательного доступа к автомобилю.

        Обработка данных в реальном времени: Обработка данных в реальном времени в беспилотных автомобилях может потребовать хранения и обработки большого объема данных на борту или в облачных сервисах, что увеличивает риск утечки или взлома данных.

        
        \item Потенциальные риски использования данных беспилотных автомобилей для слежки или дискриминации.

        Локационная слежка: Беспилотные автомобили могут собирать информацию о местоположении пользователей в реальном времени. Эти данные могут использоваться для слежки за перемещениями людей и отслеживания их привычек и поведения.

        Профилирование пользователей: Данные, собранные беспилотными автомобилями, могут использоваться для создания профилей пользователей на основе их местоположения, предпочтений поездок, частоты их перемещений и других параметров. Это может привести к дискриминации или нарушению приватности.

        Утечка личной информации: Несанкционированный доступ к данным беспилотных автомобилей или утечка данных из-за недостаточной защиты может привести к раскрытию личной информации пользователей, что может быть использовано для слежки или дискриминации.

        Биас в данных и алгоритмах: Данные, собранные беспилотными автомобилями, могут содержать биасы, основанные на различных факторах, таких как расовая или социальная принадлежность. Если эти биасы присутствуют в алгоритмах принятия решений, это может привести к дискриминации в процессе предоставления услуг или принятия решений на дороге.

        Недостаточная прозрачность и контроль: Недостаточная прозрачность в отношении того, как используются данные беспилотных автомобилей, и отсутствие механизмов контроля со стороны пользователей могут увеличить риски слежки и дискриминации.

        \item Анализы необходимость разработки четких правил и норм для обеспечения этичного сбора и использования данных беспилотных автомобилей.

        Защита приватности пользователей: Личные данные, собранные беспилотными автомобилями, могут включать в себя чувствительную информацию о перемещениях, предпочтениях и других аспектах личной жизни. Четкие правила и нормы помогут обеспечить защиту этой приватной информации и предотвратить ее злоупотребление.

        Предотвращение дискриминации и биаса: Открытые и прозрачные правила помогут предотвратить использование данных для дискриминации или формирования предвзятых алгоритмов. Это может включать в себя установление норм по обработке данных и контролю за биасами в алгоритмах принятия решений.

        Безопасность данных: Четкие правила и нормы обеспечат безопасность данных, собираемых и обрабатываемых беспилотными автомобилями, защитят их от несанкционированного доступа, утечек и взлома.

        Доверие общества к технологии: Разработка этичных стандартов и правил для сбора и использования данных беспилотных автомобилей поможет установить доверие общества к этой технологии. Это сделает ее более принятым и востребованным решением на рынке.

        Соответствие законодательству и регулятивным нормам: Четкие правила и нормы будут обеспечивать соответствие беспилотных автомобилей законодательству о защите данных, такому как GDPR в Европе или CCPA в Калифорнии, а также другим регулятивным нормам в области конфиденциальности и безопасности.

        Этические нормы и ценности: Разработка этических стандартов и норм для беспилотных автомобилей позволит внедрить в технологию общепринятые этические нормы и ценности, такие как уважение к приватности, справедливость и учет интересов всех участников дорожного движения.
    \end{itemize}
\end{itemize}


\section{Заключение}

Внедрение беспилотных автомобилей может принести значительную пользу обществу, но важно не торопиться и тщательно решать этические проблемы, связанные с этой новой технологией. Только при условии обеспечения безопасности, прозрачности, подотчетности и этичности беспилотные автомобили смогут завоевать доверие общества и реализовать свой полный потенциал.

\bibliography{bibliography}

\begin{thebibliography}{9}

    \bibitem{moralmachine}
    Moral Machine. \url{https://www.moralmachine.net/}
    
    \bibitem{ethics_self_driving}
    The Ethics of Self-Driving Cars.
    
    \bibitem{driverless_trolley}
    Driverless Cars and the Trolley Problem.
    
    \bibitem{legal_ethical_challenges}
    The Legal and Ethical Challenges of Self-Driving Cars.
    
    \bibitem{blame_self_driving}
    Who's to Blame When a Self-Driving Car Kills?
    
    \end{thebibliography}

\end{document}
